\documentclass{article}

\usepackage{amsmath, amssymb}
\usepackage{hyperref}


\title{Proposal: a zero-inflated negative binomial model for differential expression analysis in single-cell RNA-seq data}
\author{Chenghao Wang}
\date{\today}

\begin{document}

\maketitle
\section{Problem Statement}
With advances in single-cell RNA-sequencing (scRNA-seq) technologies,
research on gene expression profiles has reached the single-cell level,
yet resulting in a key task of finding differentially expressed genes
between conditions, along with the need for statistical approaches to
achieve this.

\section{Background}
One of the simplest methods for differential expression (DE) analysis is to perform
a Wilcoxon rank-sum test for each gene, which is also the default method in the widely-used
Seuarat\cite{hao2024dictionary} toolkit. Nevertheless, this method is limited to two-group
comparisons and lacks the flexibility to adjust for covariates.

Besides, a natural idea is to directly apply the linear or generalized linear model based 
methods proposed for bulk RNA-seq data, such as DESeq2\cite{love_moderated_2014}, 
edgeR\cite{mccarthy_differential_2012} and limma\cite{law_voom_2014}, 
to scRNA-seq data. But these methods cannot handle the excess zeros which are rare in
bulk RNA-seq data but common in scRNA-seq data. Aggregating the cell-level data from each subject into
pseduo-bulk data before applying these methods is a useful strategy, but it may lead to a loss of power 
and is not applicable when the number of subjects is small.

Several methods tailored to DE analysis of scRNA-seq data have been developed.
SCDE\cite{kharchenko_bayesian_2014} and scDD\cite{korthauer_statistical_2016}
are Bayesian methods based on mixture models. DEsingle\cite{miao_desingle_2018}
utilizes a zero-inflated negative binomial distribution for DE analysis.
Yet these methods cannot properly control for covariates or be generalized to multi-group
comparisons.
MAST\cite{finak_mast_2015} employs a hurdle model to fit the log-transformed expression data,
while accouting for the excess zeros. But it does not distinguish 
between technical (dropout) and biolgoical (non-expression) zeros.

\section{Data}
This method will be illustrated using the dataset by Islam et al.\cite{islam2011characterization},
which is availabe at \href{https://www.ncbi.nlm.nih.gov/geo/query/acc.cgi?acc=GSE29087}{GSE29087}.
This dataset contains 48 embryonic stems cells and 44 embryonic fibroblasts in the mouse,
sequenced by single-cell tagged reverse transcription (STRT) procedure.
The 22,936 features will be filtered, retaining the coding genes with positive counts in at least 5 cells.
We will also simulate data from zero-inflated negative binomial distributions.
The simulation settings will be designed to reflect the properties of the real dataset
by Islam et al.


\section{Proposed Approach}
The read count $Y_{gi}$ for gene $g$ in cell $i$ will be modeled using a 
zero-inflated negative binomial model, as follows:
\[
\begin{aligned}
& Y_{gi}|Z_{gi} = 0 \sim NB(s_{i}\mu_{gi}, \phi_{g}) \\
& Y_{gi}|Z_{gi} = 1 \sim constant(0) \\
& logit(P(Z_{gi} = 1)) = X_{g}\alpha_{g} \\
& log(\mu_{gi}) = X_{g}\beta_{g}
\end{aligned}
\]
$Z_{gi}$ is a latent variable indicating whether $Y_{gi}$ belongs to 
the zero-inflated component or the negative binomial component.
$s_i$ is the scale factor for normalization in cell $i$, which will be calculated using the same
method as in DEsingle\cite{miao_desingle_2018}:
\[
\begin{aligned}
& s_i = \underset{g}{\text{median}}
\frac{Y_{gi}}{\left(\displaystyle\prod_{\substack{i = 1 \\ Y_{gi}>0}}^{m} Y_{gi}\right)^{1/m_g}}\\
& m_g = \sum_{i=1}^{n} I(Y_{gi} > 0)
\end{aligned}
\]

We will use the following three procedures to estimate the paramters ($\alpha_g$, $\beta_g$, $\phi_g$) in the model:
\begin{itemize}
    \item Use a Broyden-Fletcher-Goldfarb-Shanno (BFGS) algorithm, 
    which jointly estimates all parameters. 
    \item Use a Generalized Expectation-Maximization (GEM) algorithm. In the M-step,
    the parameters are jointly updated using a Fisher's scoring algorithm.
    \item Use a GEM algorithm. In the M-step, the parameters are alternately updated using a Fisher's scoring algorithm.
\end{itemize}

Since the number of cells with read counts belonging to the negative binomial component
varies across genes, the estimation of $\phi_g$ may be less stable for some genes.
To address this issue, we will use an empirical Bayes method to shrink 
the gene-wise dispersion estimates toward the prior. Two priors will be considered:
a common dispersion across genes  
and a mean-dependent dispersion trend (fitted using a local regression).
\[
\log(\phi_g) \sim N(\log(\phi(\bar{\mu_g})), \tau^2),
\]
where $\phi$ is a constant (calculated as the mean of all gene-wise dispersions) 
or a function of $\bar{\mu_g}$ (calculated as the fitted value on the mean-dispersion trend), 
the mean of the normalized read counts ($Y_{gi}/s_i$) for gene $g$.
We will use the residuals from the fitted trend to estimate the variance, $\sigma^2$, which
arises from two sources: the sampling error and the prior \cite{wu_new_2013}.
We will simulate data from $NB(s_i\hat{\mu}_{gi}, \phi(\bar{\mu_g}))$ and
estimate $\phi$ by the method of moments. Then we will estimate 
the sampling error variance, which will be subtracted from $\sigma^2$
to obtain the prior variance, $\tau^2$.

Likelihood ratio tests and Wald tests will be used to test for DE.

\section{Evaluation Plan}
First, we will apply the proposed method to the Islam et al. dataset
and evaluate the results against their paper.
Next, we will compare the proposed method with MAST in terms of type I error control by
permuting the condition labels of the cells and assessing the distribution of the p-values.
Lastly, we will compare the proposed method with MAST with respect to ROC curves using
the simulated data.

\section{Timeline}

\begin{table}[ht]
\centering
\begin{tabular}{p{0.25\textwidth} p{0.65\textwidth}}
\hline
Period & Activities \\
\hline
Week 1 & Data preprocessing and derivation of model equations\\
Weeks 2--3 & Implementation and testing of BFGS algorithm for parameter estimation \\ 
Weeks 4--5 & Implementation and testing of GEM + Fisher' scoring algorithm for parameter estimation \\
Week 6 & Implementation of empirical Bayes framework \\
Week 7 & Data simulation and method evaluation \\
\hline
\end{tabular}
\end{table}


\bibliographystyle{unsrt}
\bibliography{ref}

\end{document}
